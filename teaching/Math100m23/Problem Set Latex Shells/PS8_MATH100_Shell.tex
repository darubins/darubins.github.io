\documentclass[11pt]{article}
%------------------------
%Packages
\usepackage[top=0.75in, bottom=1.25in, left=1in, right=1in]{geometry} 
\usepackage{amsmath,amsthm,amssymb} %this is THE math package
\usepackage{mathtools}
\usepackage{tikz}
\usepackage{graphicx}
\usepackage{enumitem}
\usepackage{fancybox}
\usepackage{hyperref}
\usepackage{varwidth}
\usepackage{mdframed}
\usepackage{mathrsfs}
%------------------------
%Fonts I use, uncomment if you like to use them.
%The first is the general font, and the second a math font
\usepackage{mathpazo}
%\usepackage{eulervm}
%------------------------
%This is so that we have standard fonts for the doublestroked symbols
%for reals, naturals etc. regardless of what font you use.
%Don't comment
\AtBeginDocument{
  \DeclareSymbolFont{AMSb}{U}{msb}{m}{n}
  \DeclareSymbolFontAlphabet{\mathbb}{AMSb}}

%----------------------------------------------
%User-defined environments
%Commented because we're not using them in this document
%The only uncommented ones are the Problem and Solution environment

% \newenvironment{theorem}[2][Theorem]{\begin{trivlist}
% \item[\hskip \labelsep {\bfseries #1}\hskip \labelsep {\bfseries #2.}]}{\end{trivlist}}
% \newenvironment{lemma}[2][Lemma]{\begin{trivlist}
% \item[\hskip \labelsep {\bfseries #1}\hskip \labelsep {\bfseries #2.}]}{\end{trivlist}}
% \newenvironment{exercise}[2][Exercise]{\begin{trivlist}
% \item[\hskip \labelsep {\bfseries #1}\hskip \labelsep {\bfseries #2.}]}{\end{trivlist}}
% \newenvironment{question}[2][Question]{\begin{trivlist}
% \item[\hskip \labelsep {\bfseries #1}\hskip \labelsep {\bfseries #2.}]}{\end{trivlist}}
% \newenvironment{corollary}[2][Corollary]{\begin{trivlist}
% \item[\hskip \labelsep {\bfseries #1}\hskip \labelsep {\bfseries #2.}]}{\end{trivlist}}
\newenvironment{problem}[2][Problem\!]{\begin{trivlist}
\item[\hskip \labelsep {\bfseries #1}\hskip \labelsep {\bfseries #2.}]}{\end{trivlist}}
%\newenvironment{sub-problem}[2][]{\begin{trivlist}
%\item[\hskip \labelsep {\bfseries #1}\hskip \labelsep {\bfseries #2}]}{\end{trivlist}}
\newenvironment{solution}{\begin{proof}[\textbf{\textit{Solution}}]}{\end{proof}}
%----------------------------------------------

%----------------------------
%User-defined notations
\newcommand{\zz}{\mathbb Z}   %blackboard bold Z
\newcommand{\qq}{\mathbb Q}   %blackboard bold Q
\newcommand{\ff}{\mathbb F}   %blackboard bold F
\newcommand{\rr}{\mathbb R}   %blackboard bold R
\newcommand{\nn}{\mathbb N}   %blackboard bold N
\newcommand{\cc}{\mathbb C}   %blackboard bold C
\newcommand{\af}{\mathbb A}   %blackboard bold A
\newcommand{\pp}{\mathbb P}   %blackboard bold P
\newcommand{\id}{\operatorname{id}} %for identity map
\newcommand{\im}{\operatorname{im}} %for image of a function
\newcommand{\dom}{\operatorname{dom}} %for domain of a function
\newcommand{\cat}[1]{\mathscr{#1}}   %calligraphic category
\newcommand{\abs}[1]{\left\lvert#1\right\rvert} %for absolute value
\newcommand{\norm}[1]{\left\lVert#1\right\rVert} %for norm
\newcommand{\modar}[1]{\text{ mod }{#1}} %for modular arithmetic
\newcommand{\set}[1]{\left\{#1\right\}} %for set
\newcommand{\setp}[2]{\left\{#1\ \middle|\ #2\right\}} %for set with a property
\newcommand{\card}[1]{\#\,{#1}} %for cardinality of a set
\newcommand{\defeq}{\overset{\text{\tiny def}}{=}}

%Re-defined notations
\renewcommand{\epsilon}{\varepsilon}
\renewcommand{\phi}{\varphi}
\renewcommand{\emptyset}{\varnothing}
\renewcommand{\geq}{\geqslant}
\renewcommand{\leq}{\leqslant}
\renewcommand{\Re}{\operatorname{Re}}
\renewcommand{\gcd}{\operatorname{GCD}}
\renewcommand{\Im}{\operatorname{Im}}
%----------------------------

\allowdisplaybreaks
 
\begin{document}
 
\title{Problem Set 8}
\author{[Your Full Name Here]\\[0.5em]
MATH 100 | Introduction to Proof and Problem Solving | Summer 2023}
\date{} 
\maketitle

%Use \[...\] instead of $$...$$

\begin{problem}{8.1}
Give an example, with an explanation, of functions for the following if you think examples exist. If you think no such example exists, prove why \begin{enumerate}
    \item [(a)] An injective but not surjective function
    \begin{solution}
    %Uncomment and WRITE YOUR SOLUTION HERE
    \end{solution}
    %----------------------------------------
    \item [(b)] Let \(A = \{a,b,c\}\). A surjective function \(f: A \to \mathcal{P}(A)\) 
    \begin{solution}
    %Uncomment and WRITE YOUR SOLUTION HERE
    \end{solution}
    %----------------------------------------
    \item [(c)] A function that is neither surjective nor injective 
    \begin{solution}
    %Uncomment and WRITE YOUR SOLUTION HERE
    \end{solution}
    %----------------------------------------
    \item [(d)] A surjective but not injective injective
    \begin{solution}
    %Uncomment and WRITE YOUR SOLUTION HERE
    \end{solution}
    %----------------------------------------
    \item [(e)] Let \(A = \{a,b,c\}\). An injective function \(f: A \to \mathcal{P}(A)\) 
    \begin{solution}
    %Uncomment and WRITE YOUR SOLUTION HERE
    \end{solution}
\end{enumerate}
\end{problem}

\newpage % Do not delete

%----------------------------------------

\begin{problem}{8.2}
Let \(A, B \) be finite sets such that \(|A| = |B| = n\). Prove by induction that there are \(n!\) bijective functions from A to B.
%----------------------------------------
\begin{solution}
%Uncomment and WRITE YOUR SOLUTION HERE
\end{solution}
%----------------------------------------

\end{problem}

\newpage %Do not delete

\begin{problem}{8.3}
 Let \(f: A \to B\) be a function and let \(X \subseteq A\) and \(Y \subseteq B\). Recall we defined the sets \begin{align*}
        f(X) & = \{ y \in Y : y=f(x) \text{ for some } x \in X\} \subseteq B\\
        f^{-1}(Y) & = \{x \in X : f(x) \in Y \} \subseteq A
\end{align*}
\begin{itemize}[itemsep=3em]
    \item[(a)] Prove that \(X \subseteq f^{-1}(f(X)) \). Give an example to show that this containment can sometimes be strict (ie \( X \subsetneq f^{-1}(f(X))\))
    \begin{solution}
    %Uncomment and WRITE YOUR SOLUTION HERE
    \end{solution}
%----------------------------------------
    \item [(b)] Make a similar conjecture and then prove it about the relationship between Y and \(f(f^{-1}(Y))\) (is one contained in the other? If so, which one?)
    %----------------------------------------
    \begin{solution}
    %Uncomment and WRITE YOUR SOLUTION HERE
    \end{solution}
%----------------------------------------
    \item [(c)] Prove that \(f: A \to B\) is injective iff for all subsets \(X \subseteq A\) we have \(X = f^{-1}(f(X)) \)
    %----------------------------------------
    \begin{solution}
    %Uncomment and WRITE YOUR SOLUTION HERE
    \end{solution}
%----------------------------------------
    \item [(d)] Make a similar conjecture as in part c and prove it about f being surjective.
    \begin{solution}
    %Uncomment and WRITE YOUR SOLUTION HERE    
    \end{solution}
%----------------------------------------

\end{itemize}
\end{problem}

\newpage  %Do not delete



\begin{center}
\textbf{Collaborators:}
%List your peers with whom you discussed the Problem Set
\end{center}
\vfill 

\begin{center}
\textbf{References:}
%List any book/website/notes that you used to write your solutions
\end{center}
\begin{itemize}
\item[$\bullet$] [Book(s): Title, Author]
\item[$\bullet$] [Online: \href{http://example.com/}{Link}]
\item[$\bullet$] [Notes: \href{http://example.com/}{Link}]
\end{itemize}

\vfill
\begin{center}
Fin.
\end{center}
\vfill

\end{document}